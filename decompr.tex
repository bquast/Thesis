\batchmode
\makeatletter
\def\input@path{{/Users/quast/Thesis/}}
\makeatother
\documentclass[a4paper]{article}\usepackage[]{graphicx}\usepackage[]{color}
%% maxwidth is the original width if it is less than linewidth
%% otherwise use linewidth (to make sure the graphics do not exceed the margin)
\makeatletter
\def\maxwidth{ %
  \ifdim\Gin@nat@width>\linewidth
    \linewidth
  \else
    \Gin@nat@width
  \fi
}
\makeatother

\definecolor{fgcolor}{rgb}{0.345, 0.345, 0.345}
\newcommand{\hlnum}[1]{\textcolor[rgb]{0.686,0.059,0.569}{#1}}%
\newcommand{\hlstr}[1]{\textcolor[rgb]{0.192,0.494,0.8}{#1}}%
\newcommand{\hlcom}[1]{\textcolor[rgb]{0.678,0.584,0.686}{\textit{#1}}}%
\newcommand{\hlopt}[1]{\textcolor[rgb]{0,0,0}{#1}}%
\newcommand{\hlstd}[1]{\textcolor[rgb]{0.345,0.345,0.345}{#1}}%
\newcommand{\hlkwa}[1]{\textcolor[rgb]{0.161,0.373,0.58}{\textbf{#1}}}%
\newcommand{\hlkwb}[1]{\textcolor[rgb]{0.69,0.353,0.396}{#1}}%
\newcommand{\hlkwc}[1]{\textcolor[rgb]{0.333,0.667,0.333}{#1}}%
\newcommand{\hlkwd}[1]{\textcolor[rgb]{0.737,0.353,0.396}{\textbf{#1}}}%

\usepackage{framed}
\makeatletter
\newenvironment{kframe}{%
 \def\at@end@of@kframe{}%
 \ifinner\ifhmode%
  \def\at@end@of@kframe{\end{minipage}}%
  \begin{minipage}{\columnwidth}%
 \fi\fi%
 \def\FrameCommand##1{\hskip\@totalleftmargin \hskip-\fboxsep
 \colorbox{shadecolor}{##1}\hskip-\fboxsep
     % There is no \\@totalrightmargin, so:
     \hskip-\linewidth \hskip-\@totalleftmargin \hskip\columnwidth}%
 \MakeFramed {\advance\hsize-\width
   \@totalleftmargin\z@ \linewidth\hsize
   \@setminipage}}%
 {\par\unskip\endMakeFramed%
 \at@end@of@kframe}
\makeatother

\definecolor{shadecolor}{rgb}{.97, .97, .97}
\definecolor{messagecolor}{rgb}{0, 0, 0}
\definecolor{warningcolor}{rgb}{1, 0, 1}
\definecolor{errorcolor}{rgb}{1, 0, 0}
\newenvironment{knitrout}{}{} % an empty environment to be redefined in TeX

\usepackage{alltt}
\usepackage{lmodern}
\usepackage{lmodern}
\usepackage[T1]{fontenc}
\usepackage[utf8]{inputenc}
\usepackage{rotating}
\usepackage{prettyref}
\usepackage{float}
\usepackage{rotfloat}
\usepackage{mathtools}
\usepackage{amsmath}

\makeatletter

%%%%%%%%%%%%%%%%%%%%%%%%%%%%%% LyX specific LaTeX commands.
\pdfpageheight\paperheight
\pdfpagewidth\paperwidth

%% Because html converters don't know tabularnewline
\providecommand{\tabularnewline}{\\}

%%%%%%%%%%%%%%%%%%%%%%%%%%%%%% Textclass specific LaTeX commands.
\newcommand{\code}[1]{\texttt{#1}}

%%%%%%%%%%%%%%%%%%%%%%%%%%%%%% User specified LaTeX commands.
\usepackage{rotating}
\usepackage{adjustbox}
\usepackage{threeparttable}
\usepackage{longtable}
\usepackage{changepage}
\usepackage{lscape}
\usepackage{pbox}
\usepackage{array}
\usepackage[style=philosophy-modern,natbib=true,backend=biber]{biblatex}
\addbibresource{/Users/quast/Thesis/bibliography.bib}

\newcolumntype{R}[2]{
    >{\adjustbox{angle=#1,lap=\width-(#2)}\bgroup}
    l
    <{\egroup}
}
\newcommand*\rot{\multicolumn{1}{R{45}{1em}}}
\newcommand*\rotnt{\multicolumn{1}{R{90}{1em}}}

\makeatother
\IfFileExists{upquote.sty}{\usepackage{upquote}}{}
\begin{document}

\title{\code{decompr}:Global Value Chain decomposition in R}

\author{Bastiaan Quast and Victor Kummritz}
\maketitle

\section*{Abstract}

Global Value Chains have become a central unit of analysis in research
on international trade. However, the complex matrix transformations
at the basis of most Value Chain indicators still constitute a significant
entry barrier to the field. The R package decompr solves this problem
by implementing the algorithms for the analysis of Global Value Chains
as R procedures, thereby simplifying the decomposition process. Two
methods for gross export flow decomposition using Inter-Country Input-Output
tables are provided. The first method concerns a decomposition based
on the classical \citet{wale36} insight. It derives the value added
origins of an industry's exports by source country and source industry,
using easily vailable gross trade data. The second method is the Wang-Wei-Zhu
algorithm which splits bilateral gross exports into 16 value added
components. These components can broadly be divided into domestic
and foreign value added in exports. Using the results of the two decompositions,
decompr provides a set of Global Value Chain indicators, such as the
now standard Vertical Specialisation ratio. This article summarises
the methodology of the algorithms, describes the format of the input
and output data, and exemplifies the usefulness of the two methods
on the basis of a simple example data set. 

\section{Introduction}

Global Value Chains (GVCs) refer to the quickly expanding internationalization
of production networks. Most goods we use nowadays consist of parts
that are sourced from different corners of the planet and are assembled
across different continents. A popular example of this development
is the iPhone, which uses inputs from at least five countries (USA,
China, Germany, Taiwan, South Korea) and is assembled in two (USA
and China). This has made GVCs a central topic in research on trade
and development policy. Both policy makers and academia increasingly
value the growth opportunities GVCs offer to global trade and, especially,
to developing countries. However, analysing this phenomenon empirically
requires complex matrix manipulations, since the relevant data is
only available in the form of gross flows. The decompr package enables
researchers with little background in matrix algebra and linear programming
to easily derive standard GVC indicators for statistical analysis.

The package uses Inter-Country Input-Output tables (ICIOs), such as
those published by the OECD and WTO (TiVA), the World Input Output
Database \citep{timmer2012world}, or national statistics bureaus,
as input. These tables state supply and demand relationships in gross
terms between industries within and across countries. For instance,
let us look at the example of the leather used in German manufactured
car seats. The ICIOs quantify the value of inputs that the Turkish
leather and textiles industry supplies to the German transport equipment
industry. The problem of these tables measuring gross trade flows,
is that they do not reveal how much of the value was added in the
supplying industry, and how much of the value was added in previous
stages of production, performed by other industries or even countries.

The Leontief decomposition of gross trade flows solves this problem
by reallocating the value of intermediate goods used by industries
to the original producers. In our example, the use of Argentinian
agricultural produce (raw hides) is subtracted from the Turkish leather
industry and added to the Argentinian agricultural industry. The Wang-Wei-Zhu
(henceforth WWZ) decomposition goes a step further by not only revealing
the source of the value added, but also breaking down exports into
different categories, according to final usage and destination. It
implements the theoretical work of \citet{wang2014quantifying}. The
main categories in this framework are listed below.
\begin{enumerate}
\item domestic value added in exports 
\item foreign value added in exports 
\item pure double counted terms 
\end{enumerate}

\subsection{Package Details}

The decompr package implements the algorithms for these decompositions
as R procedures and provides example data sets. We start by loading
the package and listing the functions.

\begin{knitrout}
\definecolor{shadecolor}{rgb}{0.969, 0.969, 0.969}\color{fgcolor}\begin{kframe}
\begin{alltt}
\hlcom{# load package}
\hlkwd{library}\hlstd{(decompr)}

\hlcom{# list functions in package}
\hlkwd{ls}\hlstd{(}\hlstr{'package:decompr'}\hlstd{)}
\end{alltt}
\begin{verbatim}
## [1] "decomp"              "leontief"            "load_tables"        
## [4] "load_tables_vectors" "wwz"
\end{verbatim}
\end{kframe}
\end{knitrout}

The R procedures are implemented as functions, the included functions
are listed below.
\begin{itemize}
\item \code{load\_tables\_vectors()}; transforms the input objects to an
object used for the decompositions (class: \code{decompr}) 
\item \code{leontief()}; takes a decompr object and applies the Leontief
decomposition 
\item \code{wwz()}; takes a decompr object and applies the Wang-Wei-Zhu
decomposition 
\item \code{decomp()}; a wrapper function which integrates the use of load\_tables\_vectors
with the various decompositions, using an argument \textit{method}
to specify the desired decomposition (default leontief) 
\end{itemize}
For legacy purposes, one depracated function is also available under
their original names (\code{load\_tables()}). In addition to this,
one example data sets is included.
\begin{itemize}
\item \code{\textit{\emph{leather}}}; a fictional three-country, three-sector
data set \footnote{load using: \code{data(leather)}}
\end{itemize}
Trade flow analysis often involves studying the development of a certain
variable (set) over time, thus taking the panel form. However, at
the decomposition level, the panel dimension is essentially a repeated
cross-section. Therefore, as a design decision, the time dimension
is not implemented in the package itself. Instead, we provide examples
of how this repetition can be implemented using a \code{for-loop}.

\prettyref{sec:data} introduces the data as it is used by the package
as well as two example data sets, after which \prettyref{sec:leontief}
and \prettyref{sec:wwz} summarise the theoretical derivations for
the two decompositions, and show how these can be performed in R using
decompr. We conclude with a discussion of potential uses and further
developments of GVC research.

\section{Data}

\label{sec:data} Two data sets are included in the package, one real
world data set and one minimal data set for demonstration purposes.
The former is the WIOD regional Inter-Country Input Output tables
for the year 2011 \citep{timmer2012world}. The latter is a fictional
3-country 3-sector data set, which we will use throughout this article
to demonstrate the usage and advantages of the decompr package.

\begin{knitrout}
\definecolor{shadecolor}{rgb}{0.969, 0.969, 0.969}\color{fgcolor}\begin{kframe}
\begin{alltt}
\hlcom{# load the data}
\hlkwd{data}\hlstd{(leather)}

\hlcom{# list the objects in the data set}
\hlkwd{ls}\hlstd{()}
\end{alltt}
\begin{verbatim}
## [1] "countries"  "final"      "industries" "inter"      "out"
\end{verbatim}
\end{kframe}
\end{knitrout}

This data is set up in order to illustrate the benefits of the decompositions.
We do this by following the flows of intermediate goods through a
fictional GVC and by showing how the readily available gross trade
flows differ from the decomposed value added flows. To this end, we
construct the elements of the input-output tables such that we have
two countries and two industries that focus on upstream tasks, which
means they focus on supplying other industries, and one country and
industry that is specialized in downstream tasks, i.e.~it serves
mainly final demand. In our example the upstream industries are Agriculture
and Textiles while the downstream industry is Transport Equipment.
Similarly, Argentina and Turkey represent upstream countries with
Germany being located downstream within this specific value chain
(see \prettyref{tab:leather}).

\begin{sidewaystable}
\caption{Example Input-Output Table: Leather}

\begin{tabular}{|c|c|c|c|c|c|c|c|c|c|c|c|c|c|c|}
\hline 
 &
 &
\multicolumn{3}{c|}{{\scriptsize{}Argentina}} &
\multicolumn{3}{c|}{{\scriptsize{}Turkey}} &
\multicolumn{3}{c|}{{\scriptsize{}Germany}} &
\multicolumn{3}{c|}{{\scriptsize{}Final Demand}} &
{\scriptsize{}Output }\tabularnewline
\hline 
\hline 
{\scriptsize{}Country } &
{\scriptsize{}Industry } &
\begin{turn}{90}
{\scriptsize{}Agriculture}
\end{turn} &
\begin{turn}{90}
{\scriptsize{}Textile and Leather}
\end{turn} &
\begin{turn}{90}
{\scriptsize{}Transport Equipment}
\end{turn} &
\begin{turn}{90}
{\scriptsize{}Agriculture}
\end{turn} &
\begin{turn}{90}
{\scriptsize{}Textile and Leather}
\end{turn} &
\begin{turn}{90}
{\scriptsize{}Transport Equipment}
\end{turn} &
\begin{turn}{90}
{\scriptsize{}Agriculture}
\end{turn} &
\begin{turn}{90}
{\scriptsize{}Textile and Leather}
\end{turn} &
\begin{turn}{90}
{\scriptsize{}Transport Equipment}
\end{turn} &
\begin{turn}{90}
{\scriptsize{}Argentina}
\end{turn} &
\begin{turn}{90}
{\scriptsize{}Turkey}
\end{turn} &
\begin{turn}{90}
{\scriptsize{}Germany}
\end{turn} &
\begin{turn}{90}
\end{turn}\tabularnewline
\hline 
{\scriptsize{}Argentina } &
{\scriptsize{}Agriculture } &
{\scriptsize{}16.1 } &
{\scriptsize{}5.1 } &
{\scriptsize{}1.8 } &
{\scriptsize{}3.2 } &
{\scriptsize{}4.3 } &
{\scriptsize{}0.4 } &
{\scriptsize{}3.1 } &
{\scriptsize{}2.8 } &
{\scriptsize{}4.9 } &
{\scriptsize{}21.5 } &
{\scriptsize{}6.1 } &
{\scriptsize{}8.4 } &
{\scriptsize{}77.7 }\tabularnewline
\hline 
{\scriptsize{}Argentina } &
{\scriptsize{}Textile.and.Leather } &
{\scriptsize{}2.4 } &
{\scriptsize{}8.0 } &
{\scriptsize{}3.2 } &
{\scriptsize{}0.1 } &
{\scriptsize{}3.2 } &
{\scriptsize{}1.6 } &
{\scriptsize{}1.2 } &
{\scriptsize{}3.9 } &
{\scriptsize{}11.5 } &
{\scriptsize{}16.2 } &
{\scriptsize{}1.9 } &
{\scriptsize{}5.1 } &
{\scriptsize{}58.3 }\tabularnewline
\hline 
{\scriptsize{}Argentina } &
{\scriptsize{}Transport.Equipment } &
{\scriptsize{}0.9 } &
{\scriptsize{}0.5 } &
{\scriptsize{}4.0 } &
{\scriptsize{}0.0 } &
{\scriptsize{}0.1 } &
{\scriptsize{}0.3 } &
{\scriptsize{}0.0 } &
{\scriptsize{}0.4 } &
{\scriptsize{}0.5 } &
{\scriptsize{}11 } &
{\scriptsize{}0.5 } &
{\scriptsize{}0.8 } &
{\scriptsize{}19.0 }\tabularnewline
\hline 
{\scriptsize{}Turkey } &
{\scriptsize{}Agriculture } &
{\scriptsize{}1.1 } &
{\scriptsize{}1.9 } &
{\scriptsize{}0.2 } &
{\scriptsize{}18.0 } &
{\scriptsize{}13.2 } &
{\scriptsize{}6.1 } &
{\scriptsize{}9.0 } &
{\scriptsize{}3.1 } &
{\scriptsize{}8.9 } &
{\scriptsize{}7.5 } &
{\scriptsize{}29.5 } &
{\scriptsize{}14.2 } &
{\scriptsize{}112.7 }\tabularnewline
\hline 
{\scriptsize{}Turkey } &
{\scriptsize{}Textile.and.Leather } &
{\scriptsize{}0.3 } &
{\scriptsize{}2.8 } &
{\scriptsize{}0.1 } &
{\scriptsize{}6.1 } &
{\scriptsize{}28.1 } &
{\scriptsize{}6.3 } &
{\scriptsize{}2.1 } &
{\scriptsize{}2.5 } &
{\scriptsize{}25.6 } &
{\scriptsize{}8.9 } &
{\scriptsize{}24.9 } &
{\scriptsize{}16.9 } &
{\scriptsize{}124.6 }\tabularnewline
\hline 
{\scriptsize{}Turkey } &
{\scriptsize{}Transport.Equipment } &
{\scriptsize{}0.0 } &
{\scriptsize{}0.1 } &
{\scriptsize{}0.3 } &
{\scriptsize{}4.1 } &
{\scriptsize{}3.2 } &
{\scriptsize{}8.9 } &
{\scriptsize{}0.2 } &
{\scriptsize{}0.0 } &
{\scriptsize{}1.8 } &
{\scriptsize{}1.2 } &
{\scriptsize{}18.5 } &
{\scriptsize{}4.9 } &
{\scriptsize{}43.2 }\tabularnewline
\hline 
{\scriptsize{}Germany } &
{\scriptsize{}Agriculture } &
{\scriptsize{}1.2 } &
{\scriptsize{}4.2 } &
{\scriptsize{}0.3 } &
{\scriptsize{}4.1 } &
{\scriptsize{}1.2 } &
{\scriptsize{}0.6 } &
{\scriptsize{}29.0 } &
{\scriptsize{}19.5 } &
{\scriptsize{}17.9 } &
{\scriptsize{}9.2 } &
{\scriptsize{}17.9 } &
{\scriptsize{}51.2 } &
{\scriptsize{}156.3 }\tabularnewline
\hline 
{\scriptsize{}Germany } &
{\scriptsize{}Textile.and.Leather } &
{\scriptsize{}1.3 } &
{\scriptsize{}1.1 } &
{\scriptsize{}0.0 } &
{\scriptsize{}3.2 } &
{\scriptsize{}4.8 } &
{\scriptsize{}2.6 } &
{\scriptsize{}5.1 } &
{\scriptsize{}29.1 } &
{\scriptsize{}24.1 } &
{\scriptsize{}7.9 } &
{\scriptsize{}10.1 } &
{\scriptsize{}38.5 } &
{\scriptsize{}127.8 }\tabularnewline
\hline 
{\scriptsize{}Germany } &
{\scriptsize{}Transport.Equipment } &
{\scriptsize{}2.1 } &
{\scriptsize{}1.4 } &
{\scriptsize{}3.0 } &
{\scriptsize{}4.1 } &
{\scriptsize{}3.1 } &
{\scriptsize{}3.9 } &
{\scriptsize{}11.3 } &
{\scriptsize{}8.1 } &
{\scriptsize{}51.3 } &
{\scriptsize{}25.1 } &
{\scriptsize{}35.2 } &
{\scriptsize{}68.4 } &
{\scriptsize{}217.0 }\tabularnewline
\hline 
\end{tabular}
\end{sidewaystable}

The first step of the analytical process is to load the input object
and create a decompr class object, which contains the data structures
for the decompositions. This step is not needed when using the \code{decomp()}
wrapper function but more on this later.

\begin{knitrout}
\definecolor{shadecolor}{rgb}{0.969, 0.969, 0.969}\color{fgcolor}\begin{kframe}
\begin{alltt}
\hlcom{# create the decompr object}
\hlstd{decompr_object} \hlkwb{<-} \hlkwd{load_tables_vectors}\hlstd{(} \hlkwc{x} \hlstd{= inter,}
                                       \hlkwc{y} \hlstd{= final,}
                                       \hlkwc{k} \hlstd{= countries,}
                                       \hlkwc{i} \hlstd{= industries,}
                                       \hlkwc{o} \hlstd{= out        )}

\hlcom{# inspect the content of the decompr object}
\hlkwd{ls}\hlstd{(decompr_object)}
\end{alltt}
\begin{verbatim}
##  [1] "A"         "Ad"        "Am"        "B"         "Bd"       
##  [6] "Bm"        "E"         "ESR"       "Efd"       "Eint"     
## [11] "Exp"       "G"         "GN"        "L"         "N"        
## [16] "Vc"        "Vhat"      "X"         "Y"         "Yd"       
## [21] "Ym"        "bigrownam" "fdc"       "i"         "k"        
## [26] "rownam"    "z"         "z01"       "z02"
\end{verbatim}
\end{kframe}
\end{knitrout}

As can be seen above, a decompr class object is in fact a list containing
thirty different objects. For example, \textit{Eint} is an object
that collects the intermediate goods exports of the industries, while
\textit{Y} refers to the final demand that the industries supply.
Depending on the choice of the decomposition, all or some of these
objects are used.

\section{Leontief decomposition}

\label{sec:leontief}Let us now turn to the algorithms, starting with
the Leontief decomposition. We shortly describe the theoretical derivation
of the method to expose the internal steps of the decompr package.
Afterwards, we turn to the technical implementation and, finally,
we describe the output.

\subsection{Theoretical derivation}

The tools to derive the source decomposition date back to \citet{wale36}
who showed that, with a set of simple calculations, national Input-Output
tables based on gross terms give the true value added flows between
industries. The idea behind this insight is that the production of
industry \textit{i}'s output requires inputs of other industries and
\textit{i}'s own value added. The latter is the direct contribution
of \textit{i}'s output to domestic value added. The former refers
to the first round of \textit{i}'s indirect contribution to domestic
value added since the input from other industries that \textit{i}
requires for its own production triggers the creation of value added
in the supplying industries. As supplying industries usually depend
on inputs from other industries, this sets in motion a second round
of indirect value added creation in the supplying industries of the
suppliers, which is also caused by \textit{i}'s production. This goes
on until value added is traced back to the original suppliers and
can mathematically be expressed as

\begin{equation}
VB=V+VA+VAA+VAAA+...=V(I+A+A^{2}+A^{3}+...),
\end{equation}
which, as an infinite geometric series with the elements of $A<1$,
simplifies to

\begin{equation}
VB=V(I-A)^{-1},
\end{equation}
where \textit{V} is a $NxN$ matrix with the diagonal representing
the direct value added contribution of $N$ industries, \textit{A}
is the Input-Output coefficient matrix with dimension $NxN$, i.e.
it gives the direct input flows between industries required for 1\$
of output, and $B=(I-A)^{-1}$ is the so called Leontief inverse.
\textit{VB} gives thus a $NxN$ matrix of so called value added multipliers,
which denote the amount of value added that the production of an industry's
1\$ of output or exports brings about in all other industries. Looking
from the perspective of the supplying industries, the matrix gives
the value added that they contribute to the using industry's production.
If we multiply it with a $NxN$ matrix whose diagonal specifies each
industry's total output or exports, we get value added origins as
absolute values instead of shares.

The application of the Leontief insight to ICIOs as opposed to national
Input-Output tables for our Leontief decomposition is straightforward.
\textit{V} refers now to a vector of direct value added contributions
of all industries across the different countries. Its dimension is
correspondingly $1xGN$, where $G$ is the number of countries. \textit{A}
is now of dimension $GNxGN$ and gives the industry flows including
cross border relationships. Since we are interested in the value added
origins of exports we multiply these two matrices with a $GNxGN$
matrix whose diagonal we fill with each industry's exports, $E$,
such that the basic equation behind the source decomposition is given
by $V(I-A)^{-1}E$. \footnote{When using the leontief\_output function, the value added multiplier
is instead multiplied with each industry's output.} In a simple example with two countries (\textit{k} and \textit{l})
and industries (\textit{i} and \textit{j}) we can zoom in to see the
matrices' content:

\begin{align}
\begin{split}V(I-A)^{-1}E & =\begin{pmatrix}v_{k}^{i} & 0 & 0 & 0\\
0 & v_{k}^{j} & 0 & 0\\
0 & 0 & v_{l}^{i} & 0\\
0 & 0 & 0 & v_{l}^{j}
\end{pmatrix}*\begin{pmatrix}b_{kk}^{ii} & b_{kk}^{ij} & b_{kl}^{ii} & b_{kl}^{ij}\\
b_{kk}^{ji} & b_{kk}^{jj} & b_{kl}^{ji} & b_{kl}^{jj}\\
b_{lk}^{ii} & b_{lk}^{ij} & b_{ll}^{ii} & b_{ll}^{ij}\\
b_{lk}^{ji} & b_{lk}^{jj} & b_{ll}^{ji} & b_{ll}^{jj}
\end{pmatrix}\\
 & *\begin{pmatrix}e_{k}^{i} & 0 & 0 & 0\\
0 & e_{k}^{j} & 0 & 0\\
0 & 0 & e_{l}^{i} & 0\\
0 & 0 & 0 & e_{l}^{j}
\end{pmatrix}\\
 & =\begin{pmatrix}v_{k}^{i}b_{kk}^{ii}e_{k}^{i} & v_{k}^{i}b_{kk}^{ij}e_{k}^{j} & v_{k}^{i}b_{kl}^{ii}e_{l}^{i} & v_{k}^{i}b_{kl}^{ij}e_{l}^{j}\\
v_{k}^{j}b_{kk}^{ji}e_{k}^{i} & v_{k}^{j}b_{kk}^{jj}e_{k}^{j} & v_{k}^{j}b_{kl}^{ji}e_{l}^{i} & v_{k}^{j}b_{kl}^{jj}e_{l}^{j}\\
v_{l}^{i}b_{lk}^{ii}e_{k}^{i} & v_{l}^{i}b_{lk}^{ij}e_{k}^{j} & v_{l}^{i}b_{ll}^{ii}e_{l}^{i} & v_{l}^{i}b_{ll}^{ij}e_{l}^{j}\\
v_{l}^{j}b_{lk}^{ji}e_{k}^{i} & v_{l}^{j}b_{lk}^{jj}e_{k}^{j} & v_{l}^{j}b_{ll}^{ji}e_{l}^{i} & v_{l}^{j}b_{ll}^{jj}e_{l}^{j}
\end{pmatrix}\\
 & =\begin{pmatrix}vae_{kk}^{ii} & vae_{kk}^{ij} & vae_{kl}^{ii} & vae_{kl}^{ij}\\
vae_{kk}^{ji} & vae_{kk}^{jj} & vae_{kl}^{ji} & vae_{kl}^{jj}\\
vae_{lk}^{ii} & vae_{lk}^{ij} & vae_{ll}^{ii} & vae_{ll}^{ij}\\
vae_{lk}^{ji} & vae_{lk}^{jj} & vae_{ll}^{ji} & vae_{ll}^{jj}
\end{pmatrix}
\end{split}
\end{align}
\begin{align*}
 & v_{c}^{s}=\frac{va_{c}^{s}}{y_{c}^{s}}=1-a_{kc}^{is}-a_{kc}^{js}-a_{lc}^{js}-a_{lc}^{is}\hspace{2cm}(c\in k,l\hspace{0.5cm}s\in i,j),\\
 & \begin{pmatrix}b_{kk}^{ii} & b_{kk}^{ij} & b_{kl}^{ii} & b_{kl}^{ij}\\
b_{kk}^{ji} & b_{kk}^{jj} & b_{kl}^{ji} & b_{kl}^{jj}\\
b_{lk}^{ii} & b_{lk}^{ij} & b_{ll}^{ii} & b_{ll}^{ij}\\
b_{lk}^{ji} & b_{lk}^{jj} & b_{ll}^{ji} & b_{ll}^{jj}
\end{pmatrix}=\begin{pmatrix}1-a_{kk}^{ii} & -a_{kk}^{ij} & -a_{kl}^{ii} & -a_{kl}^{ij}\\
-a_{kk}^{ji} & 1-a_{kk}^{jj} & -a_{kl}^{ji} & -a_{kl}^{jj}\\
-a_{lk}^{ii} & -a_{lk}^{ij} & 1-a_{ll}^{ii} & -a_{ll}^{ij}\\
-a_{lk}^{ji} & -a_{lk}^{jj} & -a_{ll}^{ji} & 1-a_{ll}^{jj}
\end{pmatrix}^{-1},
\end{align*}
and 
\[
a_{cf}^{su}=\frac{inp_{cf}^{su}}{y_{f}^{u}}\hspace{2cm}(c,f\in k,l\hspace{0.5cm}s,u\in i,j),
\]
where $v_{s}^{c}$ gives the share of industry \emph{s}'s value added,
$va_{c}^{s},$ in output, $y_{s}^{c}$, and $e_{k}^{i}$ indicates
gross exports. Finally, $a_{su}^{cf}$ denotes the share of inputs,
$inp_{su}^{cf}$, in output. The elements of the $V(I-A)^{-1}E$ or
$vae$ matrix are our estimates for the country-industry level value
added origins of each country-industry's exports. decompr implements
this algorithm into R to automate the process of deriving the matrix.
Equipped with it, researchers can calculate standard GVC indicators.
Examples include \citet{dahuetal01}'s Vertical Specialisation ratio
at the industry-level using the \code{vertical\_specialisation} function,
which sums for each country and industry across the value added of
all foreign countries and industries, and \citet{rojoguno12}'s so-called
VAX ratio. Alternatively, the four dimensions of the matrix (source
country, source industry, using country, using industry) allow for
industry-level gravity-type estimations of value added trade flows.

\subsection{Implentation}

As described, in \prettyref{sec:data}, the first step of our analytical
process is to construct a decompr object using the \code{load\_tables\_vectors()}
function. After this, we can use the \code{leontief()} function to
apply to Leontief decomposition. 

\begin{knitrout}
\definecolor{shadecolor}{rgb}{0.969, 0.969, 0.969}\color{fgcolor}\begin{kframe}
\begin{alltt}
\hlstd{lt} \hlkwb{<-} \hlkwd{leontief}\hlstd{( decompr_object )}
\end{alltt}
\end{kframe}
\end{knitrout}

In addition, a wrapper function called \code{decomp()} is provided
which integrates both elements of the workflow into a single function.
We recomended that the atomic functions be used for large data sets,
however, for small data sets this is an easy way to derive the results
immediatly. The \code{decomp()} function requires a method to be
specified (see \code{help('decomp')} for details), if none is provided,
the function will default to \code{leontief()}.

\begin{knitrout}
\definecolor{shadecolor}{rgb}{0.969, 0.969, 0.969}\color{fgcolor}\begin{kframe}
\begin{alltt}
\hlstd{lt2} \hlkwb{<-} \hlkwd{decomp}\hlstd{(} \hlkwc{x} \hlstd{= inter,}
               \hlkwc{y} \hlstd{= final,}
               \hlkwc{k} \hlstd{= countries,}
               \hlkwc{i} \hlstd{= industries,}
               \hlkwc{o} \hlstd{= out,}
               \hlkwc{method} \hlstd{=} \hlstr{"leontief"} \hlstd{)}
\end{alltt}
\end{kframe}
\end{knitrout}

Note that the output produced by these two different processes is
identical.

\subsection{Output}

We can now analyse the output of the Leontief decomposition, which
consists of a $GNxGN$ matrix that gives for each country and industry
the value added origins of its exports by country and industry. To
this end, we look at the results of the Leontief decomposition for
our example data set (\prettyref{tab:noleon}). In the first column
we find the source countries and industries while the first row contains
the using countries and industries. The first element, $28.52$, thus
gives the amount of value added that the Argentinian Agriculture industry
has contributed to the exports of the Argentinian Agriculture industry.
Similarly, the last element of this row, $4.12$, gives the amount
of value added that the Argentinian Agriculture industry has contributed
to the exports of the German Transport Equipment industry.

\begin{sidewaystable}
{\scriptsize{}\caption{Non-decomposed Values}
\label{tab:noleon} }%
\begin{tabular}{lrrrrrrrrr}
\hline 
 &
{\scriptsize{}Argentina. } &
{\scriptsize{}Argentina. } &
{\scriptsize{}Argentina. } &
{\scriptsize{}Turkey. } &
{\scriptsize{}Turkey. } &
{\scriptsize{}Turkey. } &
{\scriptsize{}Germany. } &
{\scriptsize{}Germany. } &
{\scriptsize{}Germany.}\tabularnewline
 &
{\scriptsize{}Agriculture } &
{\scriptsize{}Textile.and. } &
{\scriptsize{}Transport. } &
{\scriptsize{}Agriculture } &
{\scriptsize{}Textile.and. } &
{\scriptsize{}Transport. } &
{\scriptsize{}Agriculture } &
{\scriptsize{}Textile.and. } &
{\scriptsize{}Transport.}\tabularnewline
 &
 &
{\scriptsize{}Leather } &
{\scriptsize{}Equipment } &
 &
{\scriptsize{}Leather } &
{\scriptsize{}Equipment } &
 &
{\scriptsize{}Leather } &
{\scriptsize{}Equipment}\tabularnewline
\hline 
{\scriptsize{}Argentina.Agriculture } &
{\scriptsize{}6.88 } &
{\scriptsize{}2.49 } &
{\scriptsize{}0.25 } &
{\scriptsize{}1.30 } &
{\scriptsize{}2.04 } &
{\scriptsize{}0.08 } &
{\scriptsize{}0.77 } &
{\scriptsize{}0.68 } &
{\scriptsize{}1.76 }\tabularnewline
{\scriptsize{}Argentina.Textile.and.Leather } &
{\scriptsize{}1.03 } &
{\scriptsize{}3.91 } &
{\scriptsize{}0.44 } &
{\scriptsize{}0.04 } &
{\scriptsize{}1.52 } &
{\scriptsize{}0.31 } &
{\scriptsize{}0.30 } &
{\scriptsize{}0.95 } &
{\scriptsize{}4.13 }\tabularnewline
{\scriptsize{}Argentina.Transport.Equipment } &
{\scriptsize{}0.38 } &
{\scriptsize{}0.24 } &
{\scriptsize{}0.55 } &
{\scriptsize{}0.00 } &
{\scriptsize{}0.05 } &
{\scriptsize{}0.06 } &
{\scriptsize{}0.00 } &
{\scriptsize{}0.10 } &
{\scriptsize{}0.18 }\tabularnewline
{\scriptsize{}Turkey.Agriculture } &
{\scriptsize{}0.47 } &
{\scriptsize{}0.93 } &
{\scriptsize{}0.03 } &
{\scriptsize{}7.33 } &
{\scriptsize{}6.27 } &
{\scriptsize{}1.20 } &
{\scriptsize{}2.23 } &
{\scriptsize{}0.75 } &
{\scriptsize{}3.19 }\tabularnewline
{\scriptsize{}Turkey.Textile.and.Leather } &
{\scriptsize{}0.13 } &
{\scriptsize{}1.37 } &
{\scriptsize{}0.01 } &
{\scriptsize{}2.48 } &
{\scriptsize{}13.35 } &
{\scriptsize{}1.24 } &
{\scriptsize{}0.52 } &
{\scriptsize{}0.61 } &
{\scriptsize{}9.19 }\tabularnewline
{\scriptsize{}Turkey.Transport.Equipment } &
{\scriptsize{}0.00 } &
{\scriptsize{}0.05 } &
{\scriptsize{}0.04 } &
{\scriptsize{}1.67 } &
{\scriptsize{}1.52 } &
{\scriptsize{}1.75 } &
{\scriptsize{}0.05 } &
{\scriptsize{}0.00 } &
{\scriptsize{}0.65 }\tabularnewline
{\scriptsize{}Germany.Agriculture } &
{\scriptsize{}0.51 } &
{\scriptsize{}2.05 } &
{\scriptsize{}0.04 } &
{\scriptsize{}1.67 } &
{\scriptsize{}0.57 } &
{\scriptsize{}0.12 } &
{\scriptsize{}7.18 } &
{\scriptsize{}4.73 } &
{\scriptsize{}6.43 }\tabularnewline
{\scriptsize{}Germany.Textile.and.Leather } &
{\scriptsize{}0.56 } &
{\scriptsize{}0.54 } &
{\scriptsize{}0.00 } &
{\scriptsize{}1.30 } &
{\scriptsize{}2.28 } &
{\scriptsize{}0.51 } &
{\scriptsize{}1.26 } &
{\scriptsize{}7.06 } &
{\scriptsize{}8.65 }\tabularnewline
{\scriptsize{}Germany.Transport.Equipment } &
{\scriptsize{}0.90 } &
{\scriptsize{}0.68 } &
{\scriptsize{}0.41 } &
{\scriptsize{}1.67 } &
{\scriptsize{}1.47 } &
{\scriptsize{}0.77 } &
{\scriptsize{}2.80 } &
{\scriptsize{}1.96 } &
{\scriptsize{}18.42 }\tabularnewline
\hline 
\end{tabular}{\scriptsize \par}

{\scriptsize{}\bigskip{}
\bigskip{}
\bigskip{}
}{\scriptsize \par}

{\scriptsize{}\caption{Leontief Decomposition}
\label{tab:leon} }%
\begin{tabular}{lrrrrrrrrr}
\hline 
 &
{\scriptsize{}Argentina. } &
{\scriptsize{}Argentina. } &
{\scriptsize{}Argentina. } &
{\scriptsize{}Turkey. } &
{\scriptsize{}Turkey. } &
{\scriptsize{}Turkey. } &
{\scriptsize{}Germany. } &
{\scriptsize{}Germany. } &
{\scriptsize{}Germany.}\tabularnewline
 &
{\scriptsize{}Agriculture } &
{\scriptsize{}Textile.and. } &
{\scriptsize{}Transport. } &
{\scriptsize{}Agriculture } &
{\scriptsize{}Textile.and. } &
{\scriptsize{}Transport. } &
{\scriptsize{}Agriculture } &
{\scriptsize{}Textile.and. } &
{\scriptsize{}Transport.}\tabularnewline
 &
 &
{\scriptsize{}Leather } &
{\scriptsize{}Equipment } &
 &
{\scriptsize{}Leather } &
{\scriptsize{}Equipment } &
 &
{\scriptsize{}Leather } &
{\scriptsize{}Equipment}\tabularnewline
\hline 
{\scriptsize{}Argentina.Agriculture } &
{\scriptsize{}28.52 } &
{\scriptsize{}2.79 } &
{\scriptsize{}0.36 } &
{\scriptsize{}1.81 } &
{\scriptsize{}3.12 } &
{\scriptsize{}0.36 } &
{\scriptsize{}1.24 } &
{\scriptsize{}1.30 } &
{\scriptsize{}4.12 }\tabularnewline
{\scriptsize{}Argentina.Textile.and.Leather } &
{\scriptsize{}1.06 } &
{\scriptsize{}19.12 } &
{\scriptsize{}0.42 } &
{\scriptsize{}0.48 } &
{\scriptsize{}1.83 } &
{\scriptsize{}0.43 } &
{\scriptsize{}0.59 } &
{\scriptsize{}1.15 } &
{\scriptsize{}4.75 }\tabularnewline
{\scriptsize{}Argentina.Transport.Equipment } &
{\scriptsize{}0.21 } &
{\scriptsize{}0.14 } &
{\scriptsize{}1.06 } &
{\scriptsize{}0.03 } &
{\scriptsize{}0.08 } &
{\scriptsize{}0.04 } &
{\scriptsize{}0.02 } &
{\scriptsize{}0.07 } &
{\scriptsize{}0.19 }\tabularnewline
{\scriptsize{}Turkey.Agriculture } &
{\scriptsize{}0.72 } &
{\scriptsize{}1.34 } &
{\scriptsize{}0.12 } &
{\scriptsize{}34.93 } &
{\scriptsize{}7.00 } &
{\scriptsize{}1.48 } &
{\scriptsize{}2.55 } &
{\scriptsize{}1.52 } &
{\scriptsize{}6.18 }\tabularnewline
{\scriptsize{}Turkey.Textile.and.Leather } &
{\scriptsize{}0.41 } &
{\scriptsize{}1.39 } &
{\scriptsize{}0.12 } &
{\scriptsize{}2.69 } &
{\scriptsize{}40.17 } &
{\scriptsize{}1.32 } &
{\scriptsize{}1.11 } &
{\scriptsize{}1.15 } &
{\scriptsize{}9.51 }\tabularnewline
{\scriptsize{}Turkey.Transport.Equipment } &
{\scriptsize{}0.03 } &
{\scriptsize{}0.09 } &
{\scriptsize{}0.03 } &
{\scriptsize{}0.81 } &
{\scriptsize{}0.91 } &
{\scriptsize{}3.16 } &
{\scriptsize{}0.12 } &
{\scriptsize{}0.07 } &
{\scriptsize{}0.65 }\tabularnewline
{\scriptsize{}Germany.Agriculture } &
{\scriptsize{}0.93 } &
{\scriptsize{}2.25 } &
{\scriptsize{}0.16 } &
{\scriptsize{}2.31 } &
{\scriptsize{}2.06 } &
{\scriptsize{}0.51 } &
{\scriptsize{}29.88 } &
{\scriptsize{}5.25 } &
{\scriptsize{}9.60 }\tabularnewline
{\scriptsize{}Germany.Textile.and.Leather } &
{\scriptsize{}0.65 } &
{\scriptsize{}0.73 } &
{\scriptsize{}0.08 } &
{\scriptsize{}1.54 } &
{\scriptsize{}2.55 } &
{\scriptsize{}0.63 } &
{\scriptsize{}1.46 } &
{\scriptsize{}18.96 } &
{\scriptsize{}8.16 }\tabularnewline
{\scriptsize{}Germany.Transport.Equipment } &
{\scriptsize{}0.67 } &
{\scriptsize{}0.65 } &
{\scriptsize{}0.26 } &
{\scriptsize{}1.29 } &
{\scriptsize{}1.49 } &
{\scriptsize{}0.57 } &
{\scriptsize{}1.73 } &
{\scriptsize{}1.51 } &
{\scriptsize{}34.74 }\tabularnewline
\hline 
\end{tabular}{\scriptsize \par}
\end{sidewaystable}

A key advantage of the decomposition becomes clear when we compare
the decomposed values with the intermediate trade values of the non-decomposed
IO table when multiplied with the exports over output ratio to create
comparability (\prettyref{tab:noleon}). We see for instance that
Argentina's Agriculture industry contributes significantly more value
added to the German Transport Equipment industry than suggested by
the non-decomposed IO table. The reason is that Argentina's Agriculture
industry is an important supplier to Turkey's Textile and Leather
industry which is in turn an important supplier for the German Transport
Equipment industry. The decomposition thus allows us to see how the
value added flows along this Global Value Chain.

We can also take look at specific industries. For instance, we find
that the non-decomposed values of the Transport Equipment are for
many elements larger than the value added elements while the opposite
holds for Agriculture. This emphasises the fact that Transport Equipment
is a downstream industry that produces mostly final goods. Agriculture
on the other hand qualifies as an upstream industry that produces
also many intermediate goods so that its value added in other industries
is typically large.

Finally let's consider the countries of our specific example. We see
that Germany has more instances in which the non-decomposed values
are above the value added flows than Argentina and Turkey combined.
Along the lines of the industry analysis, this shows that Germany
focuses within this GVC on downstream tasks producing mostly final
goods that contain value added from countries located more upstream.
In our example these are Turkey and Argentina.

\section{Wang-Wei-Zhu decomposition}

\label{sec:wwz} The Wang-Wei-Zhu decomposition builds upon the Leontief
insight but uses, in addition, further valuable information provided
in ICIOs. More specifically, the Leontief decomposition traces the
value added back to where it originates but ICIOs also contain data
on how the value added is subsequently used. This information is extracted
by the Wang-Wei-Zhu decomposition, which thereby allows a much more
detailed look at the structures of international production networks
and the respective positions of countries and industries within them.

\subsection{Theoretical derivation}

The derivation of the Wang-Wei-Zhu decomposition is significantly
more technical than the source decomposition since it splits gross
exports up more finely. This is why we present here only the final
equation for a two country one industry model (equation 22 in WWZ)
and refer the interested reader to the original paper by \citet{wang2014quantifying}.
The key idea is to use the Leontief insight and extend it using additional
information from ICIOs on the final usage and destination of the exports
(e.g. re-imported vs. absorbed abroad). 
\begin{align}
\begin{split}E^{kl}= & \left(V^{k}B^{kk}\right)^{T}*F^{kl}+\left(V^{k}L^{kk}\right)^{T}*\left(A^{kl}B^{ll}F^{ll}\right)\\
+ & \left(V^{k}L^{kk}\right)^{T}*(A^{kl}\sum_{t\neq k,l}^{G}B^{lt}F^{tt})+\left(V^{k}L^{kk}\right)^{T}*(A^{kl}B^{ll}\sum_{t\neq k,l}^{G}F^{lt})\\
+ & \left(V^{k}L^{kk}\right)^{T}*(A^{kl}\sum_{t\neq k}^{G}\sum_{l,u\neq k,t}^{G}B^{lt}F^{tu})+\left(V^{k}L^{kk}\right)^{T}*\left(A^{kl}B^{ll}F^{lk}\right)\\
+ & \left(V^{k}L^{kk}\right)^{T}*(A^{kl}\sum_{t\neq k,l}^{G}B^{lt}F^{tk})+\left(V^{k}L^{kk}\right)^{T}*\left(A^{kl}B^{lk}F^{kk}\right)\\
+ & \left(V^{k}L^{kk}\right)^{T}*(A^{kl}\sum_{t\neq k}^{G}B^{lk}F^{kt})+\left(V^{k}B^{kk}-V^{k}L^{kk}\right)^{T}*\left(A^{kl}X^{l}\right)\\
+ & \left(V^{l}B^{lk}\right)^{T}*F^{kl}+\left(V^{l}B^{lk}\right)^{T}*\left(A^{kl}L^{ll}F^{ll}\right)+\left(V^{l}B^{lk}\right)^{T}\\
* & \left(A^{kl}L^{ll}E^{l*}\right)+(\sum_{t\neq k,l}^{G}V^{t}B^{tk})^{T}*F^{kl}+(\sum_{t\neq k,l}^{G}V^{t}B^{tk})^{T}\\
* & \left(A^{kl}L^{ll}F^{ll}\right)+(\sum_{t\neq k,l}^{G}V^{t}B^{tk})^{T}*\left(A^{kl}L^{ll}E^{l*}\right),
\end{split}
\label{eq:wwz}
\end{align}
where $F^{kl}$ is the final demand in $l$ for goods of $k$, $L^{ll}$
refers to the national Leontief inverse as opposed to the Inter-Country
inverse $B$, and \textit{T} indicates a matrix transpose operation.
As can be seen from equation (\ref{eq:wwz}), the Wang-Wei-Zhu decomposition
splits gross exports into 16 terms with three main categories given
by domestic value added in exports (\textit{DViX\_B}), foreign value
added in exports (\textit{FVA}), and purely double counted terms (\textit{PDC}).
The main categories are further divided according to their final destination
so that the final decomposition is given by: 
\begin{itemize}
\item Domestic value added absorbed abroad (\textit{VAX\_G}, T1-5) 

\begin{itemize}
\item Domestic value added in final exports (\textit{DVA\_FIN}, T1) 
\item Domestic value added in intermediate exports 

\begin{itemize}
\item Domestic value added in intermediate exports absorbed by direct importers
(\textit{DVA\_INT}, T2) 
\item Domestic value added in intermediate exports re-exported to third
countries (\textit{DVA\_INTrex}, T3-5) 

\begin{itemize}
\item Domestic value added in intermediate exports re-exported to third
countries as intermediate goods to produce domestic final goods (\textit{DVA\_INTrexI1},
T3) 
\item Domestic value added in intermediate exports re-exported to third
countries as final goods (\textit{DVA\_INTrexF}, T4) 
\item Domestic value added in intermediate exports re-exported to third
countries as intermediate goods to produce exports (\textit{DVA\_INTrexI2},
T5) 
\end{itemize}
\end{itemize}
\end{itemize}
\item Domestic value added returning home (\textit{RDV\_B}, T6-8) 

\begin{itemize}
\item Domestic value added returning home as final goods (\textit{RDV\_FIN},
T6) 
\item Domestic value added returning home as final goods through third countries
(\textit{RDV\_FIN2}, T7) 
\item Domestic value added returning home as intermediate goods (\textit{RDV\_INT},
T8) 
\end{itemize}
\item Foreign value added (\textit{FVA}, T11-12/14-15 ) 

\begin{itemize}
\item Foreign value added in final good exports (\textit{FVA\_FIN}, T11/14)

\begin{itemize}
\item Foreign value added in final good exports sourced from direct importer
(\textit{MVA\_FIN}, T11) 
\item Foreign value added in final good exports sourced from other countries
(\textit{OVA\_FIN}, T14) 
\end{itemize}
\item Foreign value added in intermediate good exports (\textit{FVA\_INT},
T12/15) 

\begin{itemize}
\item Foreign value added in intermediate good exports sourced from direct
importer (\textit{MVA\_INT}, T12) 
\item Foreign value added in intermediate good exports sourced from other
countries(\textit{OVA\_INT}, T15) 
\end{itemize}
\end{itemize}
\item Pure double counting (\textit{PDC}, T9-10/13/16) 

\begin{itemize}
\item Pure double counting from domestic source (\textit{DDC}, T9-10) 

\begin{itemize}
\item Due to final goods exports production (\textit{DDF}, T9) 
\item Due to intermediate goods exports production (\textit{DDI}, T10) 
\end{itemize}
\item Pure double counting from foreign source (\textit{FDC}, T13/16) 

\begin{itemize}
\item Due to direct importer exports production (\textit{FDF}, T13) 
\item Due to other countries' exports production (\textit{FDI}, T16) 
\end{itemize}
\end{itemize}
\end{itemize}
The higher resolution of the WWZ decomposition comes at the cost of
a lower dimension (source country, using country, using industry)
since the current, highly aggregated, ICIOs render a four-dimensional
decomposition unfeasible. This means that the two methods are complementary
and imply a trade-off between detail and disaggregation.

\subsection{Implementation}

As with the \code{leontief()} function, the \code{wwz()} function
also takes a decompr class object as its input, the procedure for
this is described in \prettyref{sec:data}. After having created this
decompr object, we can apply the Wang-Wei-Zhu decomposition using
the \code{wwz()} function.

\begin{knitrout}
\definecolor{shadecolor}{rgb}{0.969, 0.969, 0.969}\color{fgcolor}\begin{kframe}
\begin{alltt}
\hlstd{w} \hlkwb{<-} \hlkwd{wwz}\hlstd{(decompr_object)}
\end{alltt}
\end{kframe}
\end{knitrout}

Furthermore, it is also possible to derive the results of the Wang-Wei-Zhu
decomposition directly, using the \code{decomp()} function.

\begin{knitrout}
\definecolor{shadecolor}{rgb}{0.969, 0.969, 0.969}\color{fgcolor}\begin{kframe}
\begin{alltt}
\hlstd{w2} \hlkwb{<-}  \hlkwd{decomp}\hlstd{(} \hlkwc{x} \hlstd{= inter,}
               \hlkwc{y} \hlstd{= final,}
               \hlkwc{k} \hlstd{= countries,}
               \hlkwc{i} \hlstd{= industries,}
               \hlkwc{o} \hlstd{= out,}
               \hlkwc{method} \hlstd{=} \hlstr{"wwz"} \hlstd{)}
\end{alltt}
\end{kframe}
\end{knitrout}

Both these processes will yield the same results.

\subsection{Output}

The output when using the WWZ algorithm is a matrix with dimensions
$GNGx19$, whereby 19 consists of the 16 objects the WWZ algorithm
decomposes exports into, plus three checksums. $GNG$ represents source
country, source industry and using country whereas these terms are
slightly ambiguous here due to the complex nature of the decomposition.
More specifically, the using country can also be the origin of the
foreign value added in the exports of the source country to the using
country (see for example T11 and T12). Therefore we use the terms
exporter, exporting industry, and direct importer instead. This becomes
much clearer when we take a look at specific examples.

\prettyref{tab:wwz} shows the results for the example data. The first
column lists exporter, exporting industry, and direct importer. Note
that the value added is domestic but not necessarily created in the
exporting industry. When exporter and importer are identical, the
values are zero since there are no exports. The first row lists the
16 components of bilateral exports at the sector level and three checksums.

The first eight components relate to domestic value added of the exporting
country contained in the sectoral exports of the exporting industry
to the direct importer. For instance, the first non-zero element in
\prettyref{tab:wwz} refers to $DVA\_FIN$, or domestic value added
in final good exports. It shows that there are 5.47 units of domestic
value added in the exports of final goods from Argentina's Agriculture
industry to Turkey. In the same row the third term, \textit{DVA\_INTrexI1},
is slightly more complicated. As mentioned above, it gives the amount
of domestic Value added in intermediate exports re-exported to third
countries as intermediate goods to produce domestic final goods. In
our example this means that there are 1.14 units of domestic value
added in the intermediate exports of Argentina's Agriculture industry
to Turkey, that are re-exported by Turkey as intermediates to a third
country which produces final goods with it. Terms six to eight concern
domestic value added that eventually returns home. \textit{RDV\_FIN2}
reveals for example that there are 0.35 units of domestic value added
in the intermediate exports of Argentina's Agriculture industry to
Turkey, that Turkey re-exports as intermediates to Argentina for the
latter's final good production.

The following four terms apply to foreign value added in exports and
separate on the one hand between the origin of the foreign value added
($MVA$ vs $OVA$) and on the other hand between the type of export
(intermediate vs final good). $MVA$ describes hereby foreign value
added sourced by the exporting country from the direct importer. From
the perspective of the latter, these terms are thus part of the $RDV$
(value added returning home) share. $OVA$ in contrast sums over the
foreign value added sourced from all other countries. Going back to
the example, this means that there are 0.21 units of Turkish value
added in the final goods exports of Argentina's Agriculture industry
to Turkey.

Terms 13 to 16 collect the double counting of gross trade statistics
that occurs when goods cross borders multiple times. $DDC$ captures
double counting due to domestic value added, which is further classified
according to the type of the ultimate export (final vs intermediate
good). $MDC$ and $ODC$, on the other hand, capture double counting
due to foreign value added from either the direct importer or other
countries. For the Argentina-Turkey case this implies, for instance,
that there are 0.18 units of value added in the intermediate exports
of Turkey to Argentina which are re-exported by Argentina's Agriculture
industry to Turkey as intermediates and then again re-exported. Since
they would be part of $MVA$ twice, they are now counted once as double-counted
term.

Finally, the three checksums give total exports, total final goods
exports, and total intermediate exports. The difference between the
first and the latter two should be zero.

One interesting application of this decomposition for trade and development
uses changes over time in $FVA\_FIN$ and $FVA\_INT$. When low-wage
developing countries enter GVCs, they tend to specialize mainly in
assembly but try to gradually move up within the value chain. To illustrate
this, we can reuse the example of the iPhone. Most of the value added
in the device stems from US design and Japanese technology but it
is ultimately assembled in China. This means for China that when it
enters this GVC, its $FVA\_FIN$ starts to increase since it imports
a lot of foreign value added, assembles it, and exports a final good:
the iPhone. However, assembly itself does not contain a lot of value
added so that the benefit of China initially is small. When its technology
improves due to the interaction with Japan and the USA, it might be
able to produce actual parts of the phone, which contain more value
added. Eventually it might even be able to outsource assembly to a
cheaper country. This would imply that it seizes a larger share of
the value added, something commonly referred to as upgrading or moving
up within the value chain. In terms of the WWZ decomposition, we would
then observe that China's $FVA\_FIN$ first goes up and then starts
to decline with a simultaneous increase in $FVA\_INT$.

\section{Conclusion}

\label{sec:conclusion}GVCs describe the increasingly international
organization of production structures. As more and more regional trade
agreements come into force, which drive down trade costs and harmonize
product standards, it becomes more and more attractive for firms to
outsource certain tasks of their production lines. Research on international
trade analysing this development evolves quickly and reveals important
implications of GVCs for economic growth and competitiveness. decompr
aims at facilitating this reasearch by simplifying the calculation
of standard GVC indicators. The purpose is to accelerate the research
and, especially, to make it accesible to a wider audience.

We have designed the package using a modular structure, with an additional
user interface function for increased ease of use. The modular structure
enables users to break the computationally intensive analysis process
down into several steps. Furthermore, the modular structure enables
users and other developers to build on top of basic data structures
which are created by the \code{load\_tables\_vectors} function when
implementing other decompositions or analyses. However, a wrapper
function (\code{decomp()}) is also provided, which combines the use
of the atomic functions into one. All of this should allow users of
the package to adapt the package to their specific needs as the GVC
research progresses.

Since GVCs constitute a fairly new field of research, there are many
ways forward for its analysis. The next central step is to examine
both in theory and empirically how GVC participation affects real
economic activity. More specifically, it is very relevant to look
at how, for instance, employment and economic growth react when countries
join GVCs and what the factors are that determine a successful relationship.
From the standpoint of developing and emerging countries a very interesting
question is if GVCs simplify industrialization and the formation of
comparative advantage while high-income countries might look for an
additional push for their stagnating post-crisis economies. We hope
that decompr can play a part in this field and promote it.

\nocite{wickham2014advanced} 

\vfill{}


\section*{Colophon}

This paper was written in a combination of R \citep{core2014r} and
\LaTeX{} \citep{lamport1986document}, specifically Lua\TeX{} \citealt{hagen2005luatex},
with biblatex and biber \citep{lehman2006biblatex} for citations,
using Sweave syntax \citep{leisch2003sweave} and compiled using knitr
\citep{yihui2013knitr} and Pandoc \citep{macfarlane2012pandoc}.

\newpage{}

\printbibliography
\end{document}
