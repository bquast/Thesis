\batchmode
\makeatletter
\def\input@path{{/Users/quast/Thesis/}}
\makeatother
\documentclass[a4paper,oneside,british]{book}\usepackage[]{graphicx}\usepackage[]{color}
%% maxwidth is the original width if it is less than linewidth
%% otherwise use linewidth (to make sure the graphics do not exceed the margin)
\makeatletter
\def\maxwidth{ %
  \ifdim\Gin@nat@width>\linewidth
    \linewidth
  \else
    \Gin@nat@width
  \fi
}
\makeatother

\definecolor{fgcolor}{rgb}{0.345, 0.345, 0.345}
\newcommand{\hlnum}[1]{\textcolor[rgb]{0.686,0.059,0.569}{#1}}%
\newcommand{\hlstr}[1]{\textcolor[rgb]{0.192,0.494,0.8}{#1}}%
\newcommand{\hlcom}[1]{\textcolor[rgb]{0.678,0.584,0.686}{\textit{#1}}}%
\newcommand{\hlopt}[1]{\textcolor[rgb]{0,0,0}{#1}}%
\newcommand{\hlstd}[1]{\textcolor[rgb]{0.345,0.345,0.345}{#1}}%
\newcommand{\hlkwa}[1]{\textcolor[rgb]{0.161,0.373,0.58}{\textbf{#1}}}%
\newcommand{\hlkwb}[1]{\textcolor[rgb]{0.69,0.353,0.396}{#1}}%
\newcommand{\hlkwc}[1]{\textcolor[rgb]{0.333,0.667,0.333}{#1}}%
\newcommand{\hlkwd}[1]{\textcolor[rgb]{0.737,0.353,0.396}{\textbf{#1}}}%

\usepackage{framed}
\makeatletter
\newenvironment{kframe}{%
 \def\at@end@of@kframe{}%
 \ifinner\ifhmode%
  \def\at@end@of@kframe{\end{minipage}}%
  \begin{minipage}{\columnwidth}%
 \fi\fi%
 \def\FrameCommand##1{\hskip\@totalleftmargin \hskip-\fboxsep
 \colorbox{shadecolor}{##1}\hskip-\fboxsep
     % There is no \\@totalrightmargin, so:
     \hskip-\linewidth \hskip-\@totalleftmargin \hskip\columnwidth}%
 \MakeFramed {\advance\hsize-\width
   \@totalleftmargin\z@ \linewidth\hsize
   \@setminipage}}%
 {\par\unskip\endMakeFramed%
 \at@end@of@kframe}
\makeatother

\definecolor{shadecolor}{rgb}{.97, .97, .97}
\definecolor{messagecolor}{rgb}{0, 0, 0}
\definecolor{warningcolor}{rgb}{1, 0, 1}
\definecolor{errorcolor}{rgb}{1, 0, 0}
\newenvironment{knitrout}{}{} % an empty environment to be redefined in TeX

\usepackage{alltt}
\usepackage[T1]{fontenc}
\usepackage[latin9]{luainputenc}
\setcounter{secnumdepth}{3}
\setcounter{tocdepth}{3}
\usepackage{babel}
\usepackage{float}
\usepackage{setspace}
\usepackage[authoryear]{natbib}
\onehalfspacing
\usepackage[unicode=true,
 bookmarks=true,bookmarksnumbered=false,bookmarksopen=false,
 breaklinks=false,pdfborder={0 0 1},backref=false,colorlinks=false]
 {hyperref}
\hypersetup{pdftitle={Three Contributions in Development Economics},
 pdfauthor={Bastiaan Quast}}

\makeatletter

%%%%%%%%%%%%%%%%%%%%%%%%%%%%%% LyX specific LaTeX commands.
\pdfpageheight\paperheight
\pdfpagewidth\paperwidth


%%%%%%%%%%%%%%%%%%%%%%%%%%%%%% Textclass specific LaTeX commands.
\newcommand{\code}[1]{\texttt{#1}}

%%%%%%%%%%%%%%%%%%%%%%%%%%%%%% User specified LaTeX commands.
\usepackage[style=philosophy-modern,backend=biber]{biblatex}

\addbibresource{/Users/quast/Making-Next-Billion-Demand-Access/man/bibliography.bib}
\addbibresource{/Users/quast/MaleFemale-Income-Child-Growth/man/bibliography.bib}

\makeatother
\IfFileExists{upquote.sty}{\usepackage{upquote}}{}
\begin{document}

\title{Reproducible Contributions in Development Economics}

\author{Bastiaan Quast}

\date{16 June 2016}

\maketitle
\tableofcontents{}


\chapter*{Introduction}

\addcontentsline{toc}{chapter}{Introduction}



As mentioned in the title, methodologically I focus on making the
research in this thesis reproducible. Reproducibility is different
from replicability in that is refers to regenerating the research
results based on the same data, as opposed to replicablity, which
uses newly gathered data. 

A large part this methodology comes from the field of biostatistics
and it is best explained by one of the key figures in this field.
\begin{quotation}
The replication of scientific findings using independent investigators,
methods, data, equipment, and protocols has long been, and will continue
to be, the standard by which scientific claims are evaluated. However,
in many fields of study there are examples of scientific investigations
that cannot be fully replicated because of a lack of time or resources.
In such a situation, there is a need for a minimum standard that can
fill the void between full replication and nothing. One candidate
for this minimum standard is \textquotedblleft reproducible research\textquotedblright ,
which requires that data sets and computer code be made available
to others for verifying published results and conducting alternative
analyses.
\end{quotation}
I believe that this applies to at least the same extend an probably
more in economics. Although there are situations in which replicable
research may be conducted, specifically in experimental settings such
as those often used in behavioural economicss or in field research
using Randomised Control Trials (RCTs), there are also many cases
in which this is not possible.

In this thesis I include two chapters where the identification strategy
is based on a natural experiment. Firslty, a change in government
policy, as a results of the unconstitutional nature of the sex-based
discrimination in pension eligibility in South Africa. Secondly, the
introduction of an interface language on the South African Google
Search website, as a spillover of that translation work being done
for Botswana. In both cases I use the National Income Dynamics Study,
the most comprehensive panel data set o

n South Africa. Since it will be hard to find more relevant data,
a full replication - using new data - will prove difficult. As such,
it is all the more important for research to be as transparent as
possible, making it at least as reproducible as can be.

There are a number of way in which research can be made reproducible,
many of which are already becoming increasingly common. In addition
to this, there are several methods which are less widespread, but
nevertheless very useful.

\newpage{}

\chapter{Making the `Next Billion' Demand Access}

\newpage{}

\chapter{Male/Female Income and Child Growth}


\chapter{\protect\code{decompr}: Global Value Chain decomposition in \protect\code{R}}


\chapter{Global Value Chains in LICs}


\chapter{\protect\code{rnn}: Recurrent Neural Networks in \protect\code{R}}


\chapter*{Final Remarks}

\addcontentsline{toc}{chapter}{Final Remarks}


\appendix

\chapter{Reproducibe Research}

Here I briefly discuss the combination of existing and new methods
and tools that I used to try and make the research in this thesis
as reproducible as possible.

First and moremost, it is essential to make clear which data is being
used and in case it is primary data, how it was produced (e.g. research
instruments). Where possible, the data itself should be included.
If this is impossible to due e.g. privacy concerns of licencing issues,
a clear procedure for obtaining the data should be documented. The
data used in both South African studies is available upon request
through an online portal. In the publicly available Git project (more
details below), I include a description how to obtain the data used
through the South African Labour \& Development Research Unit .

Secondly, the computer code for producing the research results should
be made available, in case this code produces intermediate data sets,
where possible also make available.

Comment the code, use e.g. markdown type conventions, \# for section,
use \#\# for subsection. Add a header to the file containing (commented
out):
\begin{enumerate}
\item file name
\item file purpose
\item author name
\item author email
\end{enumerate}
In addition to describing the file purpose, the file name itself should
also be meaningful. Typically the code in a file performs a certain
action, as a rule, it can therefore best be described using a transitive
verb, for instance. \code{import.R} the resulting output (here the
imported data) can then be saved using an intransitive verb as a file
name, e.g. \code{imported.RData}.

Development versioning using Git, also useful for registering research
designs (since logged), retracing steps. In addition to this, changes
in all of the project files are logged using a version control software
package. I use the open-source version control software called Git.
Projects using version control such as this are called repositories.
These repositories are published on the internet on websites such
as GitHub or BitBucket.

Good reasons to use open-source R. Using open-source software in order
to make the computational procedures of the statistical method verifiable.
In my case this means that all statistical analysis has been done
using the statistical programming environment R. 

Additionally, when implementing new algorithms, packaging these as
R libraries (extensions) and releasing under an open-source licence
such as General Public Licence version 3 or later on the Comprensive
R Archive Network (CRAN) website is a step towards replicablity. For
instance, I packages the procedures implementing the Wang-Wei-Zhu
algorithm used for the analysis as the \code{decompr} package. I
uploaded this package to CRAN\footnote{Available at: https://cran.r-project.org/web/packages/decompr/index.html}
and it has since been downloaded over 6,000 times\footnote{Data from the RStudio servers only, other CRAN mirrors to not release
statistics so the actual number is presumably higher. Note that this
does not correspond directly to users, since updates require a new
download.}. This makes code useful for replicable research, since the original
procedures, which may have contained hardcoded procedures (e.g. doing
a loop 34 times, one for each country) now have to generalised and
transformed into a function, since R packages can only contain functions.
As a result of this, the code which we developed for the analysis
of the TiVa  data set, is now also being applied by other used to
the wiod data set.

Adding weaving code . This also ties into \code{packrat}, since
it makes clear which version of which libraries are used for the analysis.

\begin{table}[H]
\caption{R \protect\code{sessionInfo()}}

\begin{knitrout}
\definecolor{shadecolor}{rgb}{0.969, 0.969, 0.969}\color{fgcolor}\begin{kframe}
\begin{alltt}
\hlkwd{sessionInfo}\hlstd{()}
\end{alltt}
\begin{verbatim}
## R version 3.3.0 (2016-05-03)
## Platform: x86_64-apple-darwin13.4.0 (64-bit)
## Running under: OS X 10.11.5 (El Capitan)
## 
## locale:
## [1] C
## 
## attached base packages:
## [1] stats     graphics  grDevices utils     datasets  base     
## 
## other attached packages:
## [1] dplyr_0.4.3 knitr_1.13 
## 
## loaded via a namespace (and not attached):
##  [1] R6_2.1.2       assertthat_0.1 magrittr_1.5   formatR_1.4   
##  [5] parallel_3.3.0 DBI_0.4-1      tools_3.3.0    Rcpp_0.12.5   
##  [9] stringi_1.0-1  methods_3.3.0  stringr_1.0.0  evaluate_0.9
\end{verbatim}
\end{kframe}
\end{knitrout}
\end{table}

Piping functions to make to more intuitive. As will most lines will
have a structure beginning with the object to which the output is
assigned followed by the assignment operator, followed by the function,
followed by the input data, followed by the argument. After which
the following line again will begin with the output object, etc. We
can see this for instance, if we use the built-in data set `swiss`,
which examines fertility levels in the French speaking regions of
Switzerland in 1888. In this example I reproduce a simplified version
of an original study this data, looking at the difference in fertiliity
levels between predominantly Catholic and predominantly protestant
agricultural regions.

\begin{table}[H]
\caption{swiss}

\begin{knitrout}
\definecolor{shadecolor}{rgb}{0.969, 0.969, 0.969}\color{fgcolor}\begin{kframe}
\begin{alltt}
\hlcom{# load data}
\hlkwd{data}\hlstd{(swiss)}

\hlcom{# inspect data}
\hlkwd{str}\hlstd{(swiss)}
\end{alltt}
\begin{verbatim}
## 'data.frame':	47 obs. of  6 variables:
##  $ Fertility       : num  80.2 83.1 92.5 85.8 76.9 76.1 83.8 92.4 82.4 82.9 ...
##  $ Agriculture     : num  17 45.1 39.7 36.5 43.5 35.3 70.2 67.8 53.3 45.2 ...
##  $ Examination     : int  15 6 5 12 17 9 16 14 12 16 ...
##  $ Education       : int  12 9 5 7 15 7 7 8 7 13 ...
##  $ Catholic        : num  9.96 84.84 93.4 33.77 5.16 ...
##  $ Infant.Mortality: num  22.2 22.2 20.2 20.3 20.6 26.6 23.6 24.9 21 24.4 ...
\end{verbatim}
\end{kframe}
\end{knitrout}
\end{table}

The typical workflow is demonstrated below. We first define an \emph{output}
object for the filtered Agricultural regions, then the assignment
operator, followed by the function, followed by the \emph{input} data,
followed by the argument. The second line then starts with a \emph{new
}intermediate object, etc. The final line starts with the function,
then in last intermediate object, then the argument.

\begin{table}[H]
\caption{R without pipe}

\begin{knitrout}
\definecolor{shadecolor}{rgb}{0.969, 0.969, 0.969}\color{fgcolor}\begin{kframe}
\begin{alltt}
\hlstd{swiss_agr}     \hlkwb{<-} \hlkwd{filter}\hlstd{(swiss, Agriculture} \hlopt{>} \hlnum{50}\hlstd{)}
\hlstd{swiss_agr_grp} \hlkwb{<-} \hlkwd{group_by}\hlstd{(swiss_agr, Catholic} \hlopt{>} \hlnum{50}\hlstd{)}
\hlkwd{summarise}\hlstd{(swiss_agr_grp,} \hlkwd{mean}\hlstd{(Fertility,} \hlkwc{na.rm}\hlstd{=}\hlnum{TRUE}\hlstd{) )}
\end{alltt}
\begin{verbatim}
## Source: local data frame [2 x 2]
## 
##   Catholic > 50 mean(Fertility, na.rm = TRUE)
##           (lgl)                         (dbl)
## 1         FALSE                      65.62143
## 2          TRUE                      79.51667
\end{verbatim}
\end{kframe}
\end{knitrout}
\end{table}

In this example I now use the \code{magrittr} package, which has
the following pipe operator \code{\%>\%}. The first thing is the
input data, followed by the transforming function, followed by the
argument (criterium), this is passed to the next line where the second
transforming function is the first item, follow by its respective
argument, this is passed on to last line, where the data is summarised,
in this case by computing the mean.

\begin{table}[H]
\caption{R with pipe}

\begin{knitrout}
\definecolor{shadecolor}{rgb}{0.969, 0.969, 0.969}\color{fgcolor}\begin{kframe}
\begin{alltt}
\hlstd{swiss} \hlopt
        \hlkwd{filter}\hlstd{(Agriculture} \hlopt{>} \hlnum{50}\hlstd{)} \hlopt
        \hlkwd{group_by}\hlstd{(Catholic} \hlopt{>} \hlnum{50}\hlstd{)} \hlopt
        \hlkwd{summarise}\hlstd{(} \hlkwd{mean}\hlstd{(Fertility,} \hlkwc{na.rm}\hlstd{=}\hlnum{TRUE}\hlstd{) )}
\end{alltt}
\begin{verbatim}
## Source: local data frame [2 x 2]
## 
##   Catholic > 50 mean(Fertility, na.rm = TRUE)
##           (lgl)                         (dbl)
## 1         FALSE                      65.62143
## 2          TRUE                      79.51667
\end{verbatim}
\end{kframe}
\end{knitrout}
\end{table}

\end{document}
